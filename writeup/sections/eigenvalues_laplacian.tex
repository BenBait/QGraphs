\section{Eigenvalues of the Probabilistic Graph Laplacian}
The following definition is from \cite{SelfSimilarity}.
\begin{definition}[Spectrally Similar]
$L$ is spectrally similar to $L_0$ with functions $\phi_0$ and $\phi_1$ if
$$
U^*(L - z)^{-1}U = (\phi_0(z)H_0-\phi_1(z))^{-1}
$$

\end{definition}

If $\phi_0 \neq 0$ and we let $R(z) = \phi_1(z) / \phi_0(z)$, then 
$$
U^*(L - z)^{-1}U = (\phi_0(H_0-R(z)))^{-1}.
$$
Notice that the left hand side is conjugation of $L - z$, which we call the \textit{resolvent operator}, that results in an embedding of $L - z$ in a space of the same size as $L_0$.

This form will be help us make a connection to the Schur Complement that is defined below.

The segmented-complete graph is spectrally similar to the complete graph.
We can use this fact to leverage the work from \cite{VibrationModes} to find eigenvectors for the graph Hamiltonian.

Starting with the Laplacian of the complete graph, $L_0$, if the Laplacian of the segmented-complete graph is spectrally similar, we can write it as
$$
L = \begin{bmatrix}
L_0 & B \\
C   & D
\end{bmatrix}.
$$

\begin{definition}[Schur Complement]
The Schur complement, $L'$, is given by
$$
L' = L_0 - BD^{-1}C.
$$
\end{definition}

To help find the eigenvectors of the segmented-complete graph, it is useful to define the Schur complement of the resolvent operator or $L$:
$$
S(z) := L_0 - z - B(D-z)^{-1}C.
$$

If we let $v_0$ be the eigenvector of the complete graph, and $v_1$ be the eigenvector of the newly introduced nodes in the segmented-complete graph, then we can solve the eigenvalue equation
$$
Lv = zv;
$$
or if we expand and apply the rules of block-matrix multiplication, we solve
$$
L_0v_0 + Bv_1 = zv_0
Cv_0 + Dv_1 = zv_1.
$$
Solving for $v_1$ and assuming that $z$ is not in the spectrum of $D$, we find that
$$
v_1= -(D - z)^{-1}Cv_0,
$$
which means that $S(z)v_0=0$.
We can relate this with the eigenvalue equation of $L_0$,
$$
(L_0 - z_0)v_0 = 0,
$$
then let $z_0 = R(z)$, and write
$$
S(z) = \phi(z)(L_0 - R(z)).
$$
Notice that if we use the nonzero $\phi_0$ from the definition of spectrally similar and let $\phi = \phi_0$, then we can write
$$
U^*(L - z)^{-1}U = \frac{1}{S(z)}
$$

The following theorem from \cite{VibrationModes} helps explicitly write the eigenvectors for the segmented-complete graph's Laplacian:
\begin{theorem}
Suppose that z is not an eigenvalue of D, and not a zero of $\phi$.
Then z is an eigenvalue of M with an eigenvector v if and only if R(z) is an eigenvalue of $M_0$ with an eigenvector $v_0$, and $v = \begin{bmatrix} v_0\\ v'\end{bmatrix}$ where 
$$
v' = -(D - z)^{-1}Cv_0.
$$
This implies that there is a one-to-one map from the eigenspace of $M_0$ corresponding to $R(z)$ onto the eigenspace of $M$ corresponding to $z$
$$
v_0 \rightarrow v = T(z)v_0
$$
where $T(z)$ is called the eigenfunction extension matrix and is given by
$$
T(z) = \begin{bmatrix}
    I_0\\
    -(D - z)^{-1}C
\end{bmatrix}
$$
\end{theorem}
