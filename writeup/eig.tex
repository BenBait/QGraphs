\section{Eigenvalues of the Probabilistic Graph Laplacian}
The segmented-complete graph is spectrally similar to the complete graph.
We can use this fact to leverage the work from \cite{VibrationModes} to find eigenvectors for the graph Hamiltonian.
\begin{definition}[Spectrally Similar]
$L$ is spectrally similar to $L_0$ if
\end{definition}

Starting with the Laplacian of the complete graph, $L_0$, if the Laplacian of the segmented-complete graph is spectrally similar, we can write it as
$$
L = \begin{bmatrix}
L_0 & B \\
C   & D
\end{bmatrix}.
$$

\begin{definition}[Schur Complement]
The Schur complement, $L'$, is given by
$$
L' = L_0 - BD^{-1}C.
$$
\end{definition}

To help find the eigenvectors of the segmented-complete graph, it is useful to define the Schur complement of the resolvent operator or $L$:
$$
S(z) = L_0 - z - B(D-z)^{-1}C.
$$

If we let $v_0$ be the eigenvector of the complete graph, and $v_1$ be the eigenvector of the newly introduced nodes in the segmented-complete graph, then we can find solve the eigenvalue equation
$$
Lv = zv;
$$
or if we expand and apply the rules of block-matrix multiplication, we solve
$$
L_0v_0 + Bv_1 = zv_0
Cv_0 + Dv_1 = zv_1.
$$
Solving for $v_1$ and assuming that $z$ is not in the spectrum of $D$, we find that
$$
v_1= -(D - z)^{-1}Cv_0,
$$
which means that $S(z)v_0=0$.
We can relate this with the eigenvalue equation of $L_0$,
$$
(M_0 - z_0)v_0 = 0,
$$
then let $z_0 = R(z)$, and write
$$
S(z) = \phi(z)(M_0 - R(z)).
$$
The following theorem from \cite{VibrationModes} helps explicitly write the eigenvectors for the segmented-complete graph's Laplacian:
\begin{theorem}
Suppose that z is not an eigenvalue of D, and not a root zero of $\phi$
\end{theorem}
